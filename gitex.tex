\documentclass[12pt,a4paper]{scrartcl}
\usepackage[T1]{fontenc}

\usepackage[utf8]{inputenc}
\usepackage[ngerman]{babel}
\usepackage{amsmath}
\usepackage{amsfonts}
\usepackage{amssymb}
\usepackage{listings}
\author{\\ Gregor Häfner \\
{\small Matr. Nr.: 347677} }
\title{Kolaboratives schreiben von Sammelbänden und Aufsätzen unter Verwendung von \LaTeX ~und Git \bigskip \\ Sowie: \\ Dokumentation des GiTeX Scripts}

\begin{document}
\maketitle
\bigskip
\tableofcontents
\pagebreak

\section{Einleitung}

Dieser Aufsatz befasst sich mit der Verwendung von Git und \LaTeX \ als Hilfsmittel zum kollaborativen Schreiben. Während \LaTeX \ bereits ein weit verbreitetes Tool zur Erstellung wissenschaftlicher Texte ist, ist Git eher in der Softwareprogrammierung bekannt. Aufgrund der besonderen eigenschaft von \LaTeX -Texten \ erfüllt es jedoch alle Bedingungen um mit Hilfe von Git versioniert werden zu können. Neben der Versionierung, bietet Git jedoch auch viele Funktionen, die beim Kolaborativen Arbeiten an Software oder in diesem Fall an Texten helfen.

 Diese Arbeit befasst sich damit, Geisteswissenschaftlern die Verknüpfung dieser beiden Tools zum Kollaborativen arbeiten an Sammelbänden oder Aufsetzen nahezulegen. Dies soll mit Vorschlägen zu einfachen und doch effektiven Workflows geschehen. Da die Verwendung von Git jedoch einige kentnisse vorraussetzt, wird außerdem wird ein Skript vorgestellt, dass es Autoren ermöglicht diese Workflows ohne jegliche Kenntnis über Git zu vollziehen. Dieses, vom Autor dieses Textes geschriebene, Skript namens GiTeX ermöglicht es Autoren in den kolaborativen Arbeitsprozess einzubinden, ohne ihnen das einarbeiten in die Befehle von Git aufzuzwingen.

\section{Latex und Git}

\subsection{Latex}


\subsection{Git}


\section{GiTeX-script}



\section{Konklusion}



\end{document}
