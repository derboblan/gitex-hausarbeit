\documentclass[12pt,a4paper]{scrartcl}
\usepackage[T1]{fontenc}

\usepackage[utf8]{inputenc}
\usepackage[ngerman]{babel}
\usepackage{amsmath}
\usepackage{amsfonts}
\usepackage{amssymb}
\usepackage{listings}
\author{\\ Gregor Häfner \\
{\small Matr. Nr.: 347677} }
\title{Kolaboratives schreiben von Sammelbänden und Aufsätzen unter Verwendung von \LaTeX ~und Git \bigskip \\ Sowie: \\ Dokumentation des GiTeX Scripts}

\begin{document}
\maketitle
\bigskip
\tableofcontents
\pagebreak

\section{Einleitung}

Dieser Aufsatz befasst sich mit der Verwendung von Git und \LaTeX \ als Hilfsmittel zum kollaborativen Schreiben. Während \LaTeX \ bereits ein weit verbreitetes Tool zur Erstellung wissenschaftlicher Texte ist, ist Git eher in der Softwareprogrammierung bekannt. Aufgrund der besonderen eigenschaft von \LaTeX -Texten \ erfüllt es jedoch alle Bedingungen um mit Hilfe von Git versioniert werden zu können. Neben der Versionierung, bietet Git jedoch auch viele Funktionen, die beim Kolaborativen Arbeiten an Software oder in diesem Fall an Texten helfen.

 Diese Arbeit befasst sich damit, Geisteswissenschaftlern die Verknüpfung dieser beiden Tools zum Kollaborativen arbeiten an Sammelbänden oder Aufsetzen nahezulegen. Dies soll mit Vorschlägen zu einfachen und doch effektiven Workflows geschehen. Da die Verwendung von Git jedoch einige kentnisse vorraussetzt, wird außerdem wird ein Skript vorgestellt, dass es Autoren ermöglicht diese Workflows ohne jegliche Kenntnis über Git zu vollziehen. Dieses, vom Autor dieses Textes geschriebene, Skript namens GiTeX ermöglicht es Autoren in den kolaborativen Arbeitsprozess einzubinden, ohne ihnen das einarbeiten in die Befehle von Git aufzuzwingen.

\section{Die Programme}

\subsection{\LaTeX}

Im gegensatz zu üblichen Schreibprogrammen wie zum beispiel Mocrosoft-Word verfolgt \LaTeX \ nicht den \emph{What-you-see-is-what-you-get} (WYSIWYG) anspruch. Das heißt, dass das verändern der Schriftart oder das einfügen eines Bildes in \LaTeX \ nicht sofort sichtbar sind. \LaTeX \ ist eine sogenannte Markup-Language. Damit reiht sie sich neben beispielsweise HTML ein, welches vorrangig für Internetseiten genutzt wird. Die besonderheit ist dabei, dass man bestimmte Textelemente mit Markups versieht und ihnen somit besondere Eigenschaften zuteilt. Beispielsweise kann "XXX"\ mit dem Markup \verb+\section+~\verb+{XXX}+\ als neues Kapitel deklariert werden. Wenn die \LaTeX -Datei dann Kompiliert wird, wird "XXX"\ dann in der entstehenden PDF-Datei größer geschrieben, bekommt eine Nummerierung, einen neuen Absatz und wird in das Inhaltsverzeichnis aufgenommen. Das eigendtliche Layout einer \LaTeX -Datei, wird also durch einige Parameter bestimmt, die sich entweder im Kopf des Dokuments oder im Text selber befinden. 

Die Textproduktion ist also anders als in Word weitestgehend vom Layout getrennt. Dies hat den Vorteil, das man mit wenigen Handgriffen einen gut lesbaren und ansprechend aussehenden Text generieren kann. Anders als bei Word kann es also zu keinen versehentlich zu großen abständen, falschen Schriftarten oder schlecht formatierten Zitaten kommen. Außerdem, und diese eigenschaft macht \LaTeX \ für die Verwendung mit Git und zum kollaborativen schreiben besonders attraktiv, gibt es innerhalb des Textes keine Bezüge an anderer Stelle aufeinander. Damit ist gemeint, dass beispielsweise eine Fußnote zu einem bestimmten Wort im Text nicht am Ende der Seite steht. Zwar wird sie Dort in der kompilierten PDF-Datei angezeigt, im ursprünglichen Text ist sie jedoch direkt hinter dem Wort, auf das sich die Fußnote bezieht. Der Vorteil daran ist, dass es zu keinen Interferenzen zwischen Verschiedenen Autoren kommt auch wenn diese auf der Selben Seite arbeiten und verschiedenste Fußnoten einfügen. Als letzten Vorteil von Latex ist noch anzumerken, dass die Datein äußerst beständig sind. Sie beinhalten nur den Text und die Markup befehle. Da Word-Datein außerdem alle Informationen über das Layout beinhalten, kann es hier mit neuen Versionen des Programms zu Inkompatibilitäten kommen und inhalte der Datei können unter Umständen verloren gehen. Dies ist bei \LaTeX -Datein nahezu ausgeschlossen.

Im gegenzu zu den diversen Vorteilen bedeutet dies natürlich auch, das der Autor eines sollchen Textes zumindest einige Kentnisse über die Formatierungsbefehle in \LaTeX \ haben muss. Diese Kentnisse sind jedoch in der Heutigen Zeit des Internets schnell akquiriert und können auch erst während des arbeitens an einem ersten text mit \LaTeX \ erlernt werden. Ein kleiner nachteil, den dieses Trennen von Layout und Textproduktion außerdem mit sich bringt ist, dass man beim schreiben von Längeren Texten wohl möglich die Übersicht verliert. Dies liegt daran, dass Tabellen oder Bilder nicht als solche dargestellt werden, sondern nur eine Aneinanderreihung von Befehlen sind.

\subsection{Git}

Das open-source Programm Git dient zur Versionierung in der Softwareproduktion. Das heißt, dass sich keine gedanken mehr um den namen einer überarbeiteten Datei gemacht werden müssen sondern lediglich die neuen Änderungen mit einem neuen Commit im Repository vermerkt werden. Das Repository ist hierbei die Komplette sammlung allen dem Projekt zugehörigen Datein zu jedem iher Entwiklungsstadien. Jeder zustand den ein Projekt in seiner Laufbahn hatte kann dann wiederhergestellt werden. Neben der reinen Versionierung hat Git jedoch auch mächtige Tools um Verschiedene Entwicklungsstränge, die parralel ablaufen, wieder zusammenzuführen. Diverse Entwickler können sich also mit komplett unterschiedlichen Baustellen eines Projektes befassen und ihre Anderungen Versionieren, bevor sie diese wieder in das Hauptprojekt einfließen lassen. Diese Eigenschaft macht Git zu einem optimalen Tool, um Kollaborativ an einem Projekt zu arbeiten. 

Allerdings sollte man nicht verschweigen, das Git in seiner Mächtigkeit für einen Einsteiger in diesen Gefilden etwas übermannend sein kann. Da es ausschließlich von der Komandozeile des Computers bediet werden kann und die über 100 Befehle die es mit sich bringt erfordern etwas Einarbeitungszeit. \footnote{git ix} 
\bigskip \\
Dieser Text hat jedoch den Anspruch hat auch Geisteswissenschaftler, die in vielen Fällen eher wenig Erfahrung mit dem Umgang mit dem Terminal haben, den Umgang mit Git und \LaTeX \ zu ermöglichen. Daher sollen im folgenden sowohl ein möglichst einfach gehaltener Workflow als auch das GiTeX-Script für die Nötige Vereinfachung sorgen.

\section{GiTeX-script}



\section{Konklusion}



\end{document}
